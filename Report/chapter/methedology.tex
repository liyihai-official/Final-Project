\section{Methedology}
For solving the PDEs, we need to begin with descritizing the objects or the region we are going to 
evaluate via matrices. 
Without consdering more detail, the navie way for these transforming processes is to simply using 
the coordinates in $d$ dimension spaces and the values of function at the points to simplify the 
objects. 
One of the famous numerical methods based on such methedology is called Finite Difference 
Method (FDM \cite{FDM}), which is a fine way to investigate objects with regular shapes such as Cube.

\begin{figure}[htbp]
  \centering
  
  \caption{Figure Shows the FDM idea, If NEED}
  \label{<label>}
\end{figure}

The other type of methods are aiming to solve systems on irregular shapes, such as simulating the 
aerodynamics effects of a jet. 
The methods, Finate Elements Methods (FEM \cite{FEM}), are start with deviding the objects into 
numbers of elements, typically quadrilateral in 2D spaces and tetrahedron in 3D spaces.

\begin{figure}[htbp]
  \centering
  
  \caption{Figure Shows the FEM idea, a jet, If NEED}
  \label{<label>}
\end{figure}




\subsection{Finite Difference Methods}
\subsubsection{Pure Message Passing Parallel}
\subsubsection{Hybrid Parallel}


\subsection{Physics Informed Neural Networks}
\subsubsection{CUDA parallel}
\subsubsection{Hybrid Parallel}