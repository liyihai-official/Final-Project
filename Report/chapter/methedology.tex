\section{Methedology}

\subsection{Distritization}
To begin with descritizing the objects or the region we are going to 
evaluate via matrices. Without consdering more detail, the navie way for these transforming processes is to simply using 
the coordinates in $d=2,3$ dimension spaces and the values of function at the points to simplify the 
objects, which is a fine way to investigate objects with regular shapes such as Cube.
For the FDTD, we are using a fine generated $d$-Cube which has shape $\left\{n_i\right\}_i^{d}$, 
including the boundary conditions, the cude has nodes $\Pi_i (n_i+2)$ numbers. 
And it takes $4\Pi_i (n_i+2)$ bytes for float32 or $8\Pi_i (n_i+2)$ bytes for float64 to store in the memory.
With such setup, for equally spaced nodes, we get
\begin{equation}
  \Delta x_i = \frac{1}{n_i-1}
\end{equation}


\begin{figure}[htbp]
  \centering
  
  \caption{Figure Shows the FDM idea, If NEED}
  \label{<label>}
\end{figure}

\subsection{Finite Difference Methods}
\subsubsection{Pure Message Passing Parallel}
\subsubsection{Hybrid Parallel}


\subsection{Physics Informed Neural Networks}
\subsubsection{CUDA parallel}
\subsubsection{Hybrid Parallel}