% \documentclass[manuscript,screen,review]{acmart}
% \documentclass[sigconf, screen, final, language=english, natbib=false]{acmart}
\documentclass[manuscript, screen, review, language=english, natbib=false]{acmart}
\usepackage{subfigure}
\usepackage{algorithm, algorithmic}
\usepackage{listings}
\usepackage{color} % use color
\usepackage{pgfplots}
\usetikzlibrary{3d,arrows.meta,shapes.geometric}
\usepackage{lipsum}
\usepackage{multirow}
\usepackage{array}

\definecolor{dkgreen}{rgb}{0,0.6,0}
\definecolor{gray}{rgb}{0.5,0.5,0.5}
\definecolor{mauve}{rgb}{0.58,0,0.82}

\lstset{frame=tb,
  language=C++, % C++
  aboveskip=3mm,
  belowskip=3mm,
  showstringspaces=false,
  columns=flexible,
  basicstyle={\small\ttfamily},
  numbers=left,
  numberstyle=\tiny\color{gray},
  keywordstyle=\color{blue},
  commentstyle=\color{dkgreen},
  stringstyle=\color{mauve},
  breaklines=true,
  breakatwhitespace=true,
  tabsize=3
}

\lstdefinestyle{customCpp}{
  language=C++,                 % 使用C++语言高亮
  frame=tb,                     % 代码块上下有边框
  aboveskip=3mm,                % 代码块上方间距
  belowskip=3mm,                % 代码块下方间距
  showstringspaces=false,       % 不显示字符串中的空格
  columns=flexible,             % 列宽度为可调节
  basicstyle={\small\ttfamily}, % 基本样式为小号等宽字体
  numbers=left,                 % 行号显示在左侧
  numberstyle=\tiny\color{gray},% 行号的样式为灰色小号字体
  keywordstyle=\color{blue},    % 关键字的颜色为蓝色
  commentstyle=\color{dkgreen}, % 注释的颜色为深绿色
  stringstyle=\color{mauve},    % 字符串的颜色为紫红色
  breaklines=true,              % 代码过长时自动换行
  breakatwhitespace=true,       % 只在空格处换行
  tabsize=2,                    % 制表符宽度为4个空格
  captionpos=b,                 % 标题位置在代码块底部
  keepspaces=true,              % 保留空格
  escapeinside={\%*}{*)},       % 可以在代码中插入LaTeX
  morekeywords={constexpr, nullptr, size_type, Integer, \&, MPI_Type_create_subarray}, % 添加自定义关键字
}


\usepackage{tikz}

\tikzstyle myBG=[line width=3pt,opacity=1.0]
\newcommand{\myGlobalTransformation}[2]
{
  \pgftransformcm{1}{0}{0.5}{0.3}{\pgfpoint{#1em}{#2em}}
}

\newcommand{\gridThreeD}[3]
{
  \begin{scope}
    \myGlobalTransformation{#1}{#2};
    \draw [#3,step=1.5em] grid (12,12);
  \end{scope}
}

\newcommand{\drawLinewithBG}[2]
{
  \draw[white,myBG]  (#1) -- (#2);
  \draw[dashed, black,thick] (#1) -- (#2);
}
\newcommand{\graphLinesHorizontal}
{
  \drawLinewithBG{1,1}{7,1};
  \drawLinewithBG{1,2.5}{7,2.5};
  \drawLinewithBG{1,4}{7,4};
  \drawLinewithBG{1,5.5}{7,5.5};
  \drawLinewithBG{1,7}{7,7};
}

\newcommand{\graphLinesVertical}
{
  \pgftransformcm{0}{1}{1}{0}{\pgfpoint{0em}{0em}}
  \graphLinesHorizontal;
}

\newcommand{\graphThreeDnodes}[2]
{
  \begin{scope}
    \myGlobalTransformation{#1}{#2};
    \foreach \x in {1,2.5,4,5.5,7} 
    {
      \foreach \y in {1,2.5,4,5.5,7} 
      {
\node at (\x,\y) [circle,fill=black,scale=0.45] {};
      }
    }
  \end{scope}
}

%% \BibTeX command to typeset BibTeX logo in the docs
\AtBeginDocument{%
  \providecommand\BibTeX{{%
    \normalfont B\kern-0.5em{\scshape i\kern-0.25em b}\kern-0.8em\TeX}}}



\setcopyright{none}

\copyrightyear{2024}
\acmYear{2024}
\acmDOI{}
\acmISBN{}

\acmConference[MAP55640]{Final Project}{May 25, 2024}{Dublin, Ireland}

% Bibliography style
\RequirePackage[
  datamodel=acmdatamodel,
  style=acmnumeric,
  ]{biblatex}

% %% Declare bibliography sources (one \addbibresource command per source)
\addbibresource{reference.bib}





\begin{document}
\title[Final Project]{MAP55640 - 
Comparative Analysis of Hybrid-Parallel Finite Difference Methods on Multicore Systems 
and 
Physics-Informed Neural Networks on CUDA Systems}


\author{Li Yihai}
\email{liy35@tcd.ie}
% \authornote{}
\affiliation{%    
    \institution{Mathematics Institute}
    \institution{High Performance Computing}
    \city{Dublin}
    \country{Ireland}
}

\author{Mike Peardon}
\authornote{Supervisor of this Final Project}
\email{mjp@maths.tcd.ie}
\affiliation{    
  \institution{Mathematics Institute}
  \institution{High Performance Computing}
  \city{Dublin}
  \country{Ireland}
}


\settopmatter{printacmref=false}


\begin{abstract}
  This is the abstract of this project.

  \lipsum[2-4]
\end{abstract}
    
% Keywords. The author(s) should pick words that accurately describe
% the work being presented. Separate the keywords with commas.
\keywords{
    Keywords
}

%%
%% This command processes the author and affiliation and title
%% information and builds the first part of the formatted document.
\maketitle

% ------------------------------------------------------------------------------------------------

\section{Introduction}


\subsection{Related Work}
\begin{frame}
  \frametitle{Introduction}
  \framesubtitle{Recap - Deep Neural Network}
  \begin{block}{Kolmogorov PDEs}
    Solving $u(x,T)$, for 
    $\mathbb{R}^1\owns T>0$, 
    $x\in \mathbb{R}^d$, 
    $t\in [0,T]$,
    $\:u(t,x)=u \in \mathbb{R}^1$,
    $ \mu(x) \in \mathbb{R}^d$,
    $\sigma(x)\in \mathbb{R}^{d\times d}$,
    \begin{equation}
      u_t
      = \frac{1}{2}
      \Tr_{\mathbb{R}^d}\left[\sigma(x)\left[\sigma(x)\right]^*\Hess_x u\right] + 
      \left< \mu(x), \nabla_xu \right>_{\mathbb{R}^d}
    \end{equation}
  \end{block}
  \begin{figure}[htbp]
    \centering
      \begin{tikzpicture}[
        % Define styles
        neuron/.style={circle, draw, fill=blue!20, minimum size=1cm},
        arrow/.style={-Stealth},
        output/.style={circle, draw, fill=green!20, minimum size=1cm},
        scale=0.9,
        transform shape
    ]
        % Neuron
        \steporfull<1->{
          \node[] (input1) at (-1,-1.2) {$u(x,T)$};
        }

        \steporfull<1->{
          \node[] (neuron) at (-1,1) {$\mathbb{E} \left[\phi (X_{T}^{x})\right]$};
          \draw[arrow] (input1) -- (neuron) node[midway, above=0.5em, left=0.5em] {\itshape Feynman-Kac};
        }
        
        % \steporfull<1->{
          \node[right=5em of neuron] (input2) {
            $\displaystyle 
              \inf_{v \in C([a,b]^d,\mathbb{R})} 
              \mathbb{E}\left[ 
                \left| \phi(X_T^x) - v(\xi) \right|^2 
              \right]$
          }; % 使用 output 样式
          \draw[arrow] (neuron) -- (input2) node[midway, above=.5em] {\itshape Prop. 2.7};
        % }
        % % \itshape Discretization \& Th. 2.8
        % \steporfull<1->{
          % \draw[-{Latex[length=2mm,width=1.5mm]}] (input2.east) arc (0:70:6mm) node[midway, right] {};
          \draw[-{Latex[length=2mm,width=1.5mm]}] 
          ([xshift=6em]input2.south) 
          arc (-120:120:6mm) 
          node[midway, right] {\itshape Discretization \& Th. 2.8};
        % }

        % \steporfull<2->{
          
        % }
      
        % \steporfull<2->{
          \node[output, right=10em of input1] (output) {$\mathbb{U}(x; \theta)$};
          \node[right=5em of output] (input3) {Neural Network};
          \draw[arrow] (input3) -- (output) node[midway, below=.5em] {\itshape Prediction};
          \draw[arrow] (input2) -- (output) node[midway, right=.5em] {\itshape Loss Function};
        % }
        
        % \steporfull<2->{
          \draw[arrow] (output) -- (input1) node[midway, below=.5em] {\itshape $\approx$ to Solution};
        % }
    \end{tikzpicture}
    \caption{Deep Neural Network (DNN) Methodology of Solving Kolmogorov PDEs \cite{FIRST}}
    \label{<label>}
  \end{figure}
  

\end{frame}

\begin{frame}
  \frametitle{Introduction}
  \framesubtitle{Recap - Physics Informed Nerual Network}
  \begin{block}{General Form of PDEs}
    % \vspace{.5em}
    -- $u(t,x)$ denotes with the target function, $x\in \mathbb{R}^{d}$. \vspace{.5em}\\
    -- $\varGamma \left[\:\:\cdot\:\:; \lambda \right]$ is a non-linear operator parameterized by $\lambda$.
    \vspace{-.5em}
    \begin{equation}
      u_t(t,x) + \varGamma \left[u; \lambda \right] = 0
    \end{equation}
    \vspace{-1.5em}
      % \vspace*{1em}
      Define $f(t, x)$ to be given by 
      % \vspace*{1em}
      \begin{equation}
        f(t,x) = u_t(t,x) + \varGamma \left[u; \lambda \right]
      \end{equation}
      % \vspace*{.2em}
  \end{block}
  \begin{figure}[htbp]
    \centering
    \begin{tikzpicture}[
      % Define styles
      neuron/.style={circle,draw,minimum size=.2cm},
      func/.style={rectangle,draw,minimum size=.2cm},
      PDE/.style={func, fill=green!50},
      loss/.style={func, fill=blue!50},
      input/.style={neuron, fill=green!50},
      hidden/.style={neuron, fill=blue!50},
      output/.style={neuron, fill=red!50},
      diff/.style={neuron, fill=green!50},
      arrow/.style={->,>=stealth},
      scale=0.85,
      transform shape
    ]
      
    \steporfull<3->{
      % Input Layer
      \draw[dashed, thick] (-1,0) rectangle (5.3,-3); % 前两个坐标为矩形左下角的坐标,后两个坐标为矩形右上角的坐标
      \node at (-1,0) [below right] {$NN(t,x; \theta)$}; % 文本内容为"文本",位置为方框的左上角

      \node[input] (Input0) at (0,-1) {$x$};
      \node[input] (Input) at (0,-2) {$t$};
    
      % Hidden Layer 1
      \node[hidden] (Hidden11) at (1.5,-0.5) {$\sigma$};
      \node[hidden] (Hidden12) at (1.5,-1.5) {$\sigma$};
      \node[hidden] (Hidden13) at (1.5,-2.5) {$\sigma$};
    
      % Hidden Layer 2
      \node[hidden] (Hidden21) at (2.5,-0.5) {$\sigma$};
      \node[hidden] (Hidden22) at (2.5,-1.5) {$\sigma$};
      \node[hidden] (Hidden23) at (2.5,-2.5) {$\sigma$};
    
      % Hidden Layer 3
      \node[hidden] (Hidden31) at (3.5,-0.5) {$\sigma$};
      \node[hidden] (Hidden32) at (3.5,-1.5) {$\sigma$};
      \node[hidden] (Hidden33) at (3.5,-2.5) {$\sigma$};

      % Output Layer
      \node[output] (Output) at (4.8,-1.5) {$u$};

      % Connect neurons Input-Hidden Layer 1
      \foreach \i in {1,2,3}
          \draw[arrow] (Input) -- (Hidden1\i);
        \foreach \i in {1,2,3}
          \draw[arrow] (Input0) -- (Hidden1\i);
    
      % Connect neurons Hidden Layer 1-Hidden Layer 2
      \foreach \i in {1,2,3}
          \foreach \j in {1,2,3}
              \draw[arrow] (Hidden1\i) -- (Hidden2\j);
    
      % Connect neurons Hidden Layer 2-Hidden Layer 3
      \foreach \i in {1,2,3}
          \foreach \j in {1,2,3}
              \draw[arrow] (Hidden2\i) -- (Hidden3\j);
    
      % Connect neurons Hidden Layer 3-Output
      \foreach \i in {1,2,3}
          \draw[arrow] (Hidden3\i) -- (Output);
    % }

    % \steporfull<2->{
      \draw[dashed, thick] (5.5,0.2) rectangle (11,-3.15);
      \node at (9.5,0.2) [below right] {PDE($\lambda$)}; % 文本内容为"文本",位置为方框的左上角
    %   % Partial Derivatives
      \node[diff] (D1) at (6.5,-0.4) {$\frac{\partial}{\partial t}$};
      \node[diff] (D2) at (7,-1.5) {$\frac{\partial}{\partial x}$};
      \node[diff] (D3) at (6.5,-2.5) {$\frac{\partial^2}{\partial^2 x}$};
      \node[PDE] (PDE) at (9.3,-1.5) {$u_t(t,x) + \varGamma \left[u; \lambda \right]$};

      \foreach \i in {1,2,3}
          \draw[arrow] (Output) -- (D\i);
      \foreach \i in {1,2,3}
          \draw[arrow] (D\i) -- (PDE);
    }

    % \steporfull<2->{
      \node[loss] (L) at (13,-1.5) {Loss};

      \draw[arrow] (Output) |- (13,-3.3) -- (L.south);
      \draw[arrow] (PDE.east) |- (L);
    % }
    \end{tikzpicture}
    \caption{PINN, with with 3 fully connected hidden layers}
    \label{}
\end{figure}
\end{frame}



\begin{frame}
  \frametitle{Introduction}
  \framesubtitle{Recap - Conclusion}
  \begin{block}{Comparing With Finite Difference Time Domain Method (FDTD)}
    \begin{enumerate}
      \item Deep Neural Network \cite{FIRST}
            \begin{itemize}
              \item Gives lower quality approximations.
              \item Takes longer time to train.
              \item Possible to solve high dimension PDEs \vspace*{1em}
            \end{itemize}
      \item Physics Informed Neural Network
            \begin{itemize}
              \item Gives higher quality approximations.
              \item Takes longer time to train.         
              \item Has more flexible way to get results.
              \item Possible to solve high dimension PDEs
            \end{itemize}
    \end{enumerate}    
  \end{block}
\end{frame}


% \subsection{Work Directions}
% \begin{frame}
%   \frametitle{Introduction}
%   \framesubtitle{Work Directions}
%   \begin{enumerate}
%     \item Implement FDTD and PINN in \texttt{C++/C}
%     \item Implement FDTD hybrid parallel version using \texttt{MPI/OpenMP}
%     \item Implement PINN GPU parallel version using \texttt{Libtorch/CUDA}
%     \item Performance optimization.
%     \item Running benchmarks
%   \end{enumerate}
% \end{frame}



\section{Related Work}
% 在工程学、物理学、生物学和化学等众多领域,为了解决各种各样的偏微分方程,学术界尝试了非常多的途径,诞生了很多数值解法并且都已经非常成熟。
% 有限差分法是一种被广泛运用的方法,通过差商来近似所需要的连续函数的导数。

To gain well quality solution of various types of PDEs is prohibitive and notoriously challanging.
The number of methods avaliable to determine canonical PDEs is limited as well,
includes 
separation of variables, 
superposition, 
product solution methods, 
Fourier transforms, 
Laplace transforms and 
perturbation methods, 
among a few others.
Even though there methods are exclusively well-performed on constrained conditions,
such as regular shaped geometry domain, constant coefficients, well-symmetric conditions 
and many others.
These limits strongly constrained the range of applicability of numerical techniques for solving PDEs,
rendering them nearly irrelevant for solving problems pratically.

General, the methods of determining numerical solutions of PDEs can be broadly classified into
two types: 
deterministic 
and stochastic. 
The mostly widely used stochastic method for solving PDEs is 
Monte Carlo Method \cite{Monte Carlo Method} which is a popular method in solving PDEs in higher dimension space with 
notable complexity.

\subsection{Finite Difference Method}
The Finithe Difference Method(FDM) is based on the numerical approximation method in calculus of finite differences.
The motivation is quiet straightforward which is approximating solutions by finding values satisfied PDEs on a set of 
presctibed interconnected points within the domain of it. Those points are which referd as nodes, and the set of nodes 
are so called as a grid of mesh.
A notable way to approximate derivatives are using 
Taylor Series expansions.
Taking 2 dimension Possion Equation as instance, assuming the investigated value as, $\varphi$,
\begin{equation}\label{EQ_POSSION_2D}
  \frac{
    \partial ^ 2 \varphi  
  }{
    \partial x ^ 2
  } +
  \frac{
    \partial ^ 2 \varphi  
  }{
    \partial y ^ 2
  }
  =  f(x,y)
\end{equation}
The total amount of nodes is denoted with $N = 15$, which gives the numerical equation which governing 
equation \ref{EQ_POSSION_2D} 
shown in 
equation \ref{EQ_POSSION_2D_10_NODES} and nodes layout as shown in the 
figure \ref{FIG_POSSION_2D_10_NODES}
\begin{equation}\label{EQ_POSSION_2D_10_NODES}
  \frac{
    \partial ^ 2 \varphi_i
  }{
    \partial x_i ^ 2
  } +
  \frac{
    \partial ^ 2 \varphi_i
  }{
    \partial y_i ^ 2
  }
  =  f(x_i,y_i) = f_i, \:\:\:\:\: i = 1,2,\dots,15
\end{equation}
\begin{figure}[htbp]
  \centering
  
  \caption{The Schematic Representa of of a 2D Computiatioal Domain and Grid. The nodes are used for the FDM by solid circles. 
  Nodes $11-15$ denote boundary nodes, while nodes $1-10$ denote internal nodes.}
  \label{FIG_POSSION_2D_10_NODES}
\end{figure}
In this case, we only need to find the value of internal nodes which $i$ is ranging from $1$ to $10$.
Next is aimming to solve this linder system \ref{EQ_POSSION_2D_10_NODES}.


\subsection{Physics Informed Neural Networks}
With the explosive growth of avaliable data and computing resouces, 
recent advances in machine learning and data analytics have yieled good results across science discipline, 
including Convolutional Neural Networks (CNNs) \cite{CNN}
for image recoginition, 
Generative Pre-trained Transformer (GPT) \cite{GPT}
for natual language processing and 
Physics Informed Neural Networks (PINNs) \cite{PINN}
for handling science problems with high complexity.
PINNs is a type of machine learning model makes full use of the benefits from 
Auto-differentiation (AD) \cite{AD}
which led to the emergence of a subject called 
Matrix Calculus \cite{Matrix_Calculus}.
Considering the parametrized and nonlinear PDEs of the general form [E.q. \ref{EQ_General_PDEs}] of function $u(t,x)$
\begin{equation}\label{EQ_General_PDEs}
  u_t + \mathcal{N}\left[u;\lambda\right] = 0
\end{equation}
The $\mathcal{N}[\cdot;\lambda]$ is a nonlinear operator which parametrized by $\lambda$.
This setup includes common PDEs problems like heat equation, and black-stokz equation and others.
In this case, we setup a neural network $NN[t,x;\theta]$ which has trainable weights $\theta$ and takes 
$t$ and $x$ as inputs, outputs with the predicting value $\hat{u}(t,x)$.
In the training process, the next step is calculating the necessary derivatives of $u$ with the respect to $t$ and $x$.
The value of loss function is a combination of the metrics of how well does these predictions fit the given conditions and 
fit the natural law [Fig. \ref{FIG_Schematic_View_PINN}]. 
\begin{figure}[htbp]
  \centering
  
  \caption{The Schematic Representa of a structure of PINN.}
  \label{FIG_Schematic_View_PINN}
\end{figure}
\section{Problem Setups}
In this project, I chose to use various numerical approaches to approximate the solutions.
It begins with Finite Difference Methods(FDM) and an other method employed by deep neural 
networks which leverage by their capability as universal function approximators
\cite{DNN-HORNIK1989359}. 

The Kormoglov PDEs are series of equations which describe the motions of Brownian Motions
\cite{SolveKorPDE}.
In general, 
let $T \in (0, +\infty)$, $d \in \mathbb{N}$, 
let $\mu : \mathbb{R}^d \rightarrow \mathbb{R}^d$ 
and $\sigma: \mathbb{R}^d \rightarrow \mathbb{R}^{d \times d}$ be the Lipschitz continuous fucntions.
Let $\varphi : \mathbb{R}^d \rightarrow \mathbb{R}$ be a function, 
and $u$ be a function from Hilbert Space 
$[0, T] \times \mathbb{R}^d$ to $\mathbb{R}$
\begin{align*}
  u : [0, T] \times \mathbb{R}^d & \longrightarrow \mathbb{R} \\
      (t, x) & \longmapsto u 
\end{align*}
with at most polynomially growing partial derivatives.
The problem is that $u$ satisfied a below system on $D = [0, T] \times \left[a, b \right]^d $ and 
$a, b \in \mathbb{R}^d$ with $a < b$,
\begin{align}
  & \frac{\partial u}{
    \partial t} 
  = \frac{1}{2}
  \text{Trace}_{\mathbb{R}^d}
  \left[
    \sigma(x) \sigma^{*}(x)
    \left(\text{Hess}_x u\right)
  \right]
  +
  \langle 
    \mu, 
    \nabla _x u
  \rangle_{
    \mathbb{R}^d
  } \\
 & u(0, x) = \varphi(x), 
 \:\:\:\: 
 x \in \left[a, b \right]^d
 \\
 & \left.
 \left(
  \frac{\partial u}{\partial \vec{n}}  + \lambda u 
 \right)
 \right|_{x \in \varGamma}
 = g(t, x) 
 \:\:\:\: 
 \forall t \in [0, T],\: x \in \mathbb{R}^d
\end{align}

The goal is to numerically appriximate the stablized state $u(T, x)$ of the system in the fureture time $T$,
and in the hypercude space $\left[a, b\right]\in \mathbb{R}^d$ with various boundary conditions $g$.

Reprompt problem, this work aimed at solving this problem in a bigger picture which requires a more general 
form. 
In this project, I consider the parametrized and nonliner PDEs of the general form
\begin{align}
  \frac{\partial u}{\partial t}
  + \mathcal{N} 
  \left[
    u; \lambda
  \right] 
  = 0
  \:\:\:\: 
  t \in \left[0. T\right], x \in D
\end{align}
where $\mathcal{N}\left[\cdot; \lambda \right]$ stands for a nonlinear operator parametrized by $\lambda$.

\subsection{Navior-Stokz Equation}
\section{Methedology}
For solving the PDEs, we need to begin with descritizing the objects or the region we are going to 
evaluate via matrices. 
Without consdering more detail, the navie way for these transforming processes is to simply using 
the coordinates in $d$ dimension spaces and the values of function at the points to simplify the 
objects. 
One of the famous numerical methods based on such methedology is called Finite Difference 
Method (FDM \cite{FDM}), which is a fine way to investigate objects with regular shapes such as Cube.

\begin{figure}[htbp]
  \centering
  
  \caption{Figure Shows the FDM idea, If NEED}
  \label{<label>}
\end{figure}

The other type of methods are aiming to solve systems on irregular shapes, such as simulating the 
aerodynamics effects of a jet. 
The methods, Finate Elements Methods (FEM \cite{FEM}), are start with deviding the objects into 
numbers of elements, typically quadrilateral in 2D spaces and tetrahedron in 3D spaces.

\begin{figure}[htbp]
  \centering
  
  \caption{Figure Shows the FEM idea, a jet, If NEED}
  \label{<label>}
\end{figure}




\subsection{Finite Difference Methods}
\subsubsection{Pure Message Passing Parallel}
\subsubsection{Hybrid Parallel}


\subsection{Physics Informed Neural Networks}
\subsubsection{CUDA parallel}
\subsubsection{Hybrid Parallel}
\section{Experiments}
\subsection{Experimental Setup}


I verify the proposed PINN methods on generated dataset, and FDTD methods with hybrid and pure MPI strategies
on the server with two Intel (R) Xeon (R) Platinum 9242 CPU nodes (96 cores per node) and $4$ NUMA nodes per node.
While the dataset is using \texttt{std::m19937} STL random device with given seed $42$ \cite{STL:RANDOM_SEED}.
The PINN models I mentioned are trained using train-from-scratch strategy, and maximin number of training times is $1'000'000$ epochs.
In the setting of learning rate and optimizer, I chose to use Adam with constant learning speed $10^{-3}$.
The details of PDEs are determined in previous section \ref{SEC:Specific_Form}.  

\paragraph{Compiling}
Compiling the program is also a critical important processes.
I chose to use the macros for defining the scale of problems in advance, this is because the compiler will 
have more aggressive optimizations if it knows more predefined parameters.

\paragraph{Running}
For finely manipulate the resource allocation on cluster, following command line \ref{LST:mpirun} is used for this 
the script arguments \texttt{rsc} stands for the resource type such as \texttt{numa}, \texttt{node} and \texttt{socket},
\texttt{--report-bindings} is a error message, for showing the details of threads and CPUs tasks allocated on cluster.
\begin{lstlisting}[
  style=customCpp,
  caption={main command line for launching program on cluster},
  label={LST:mpirun}]
    mpirun --map-by ppr:$ppr:$rsc:pe=$threads --report-bindings <executable> <arguments>
\end{lstlisting}
and \texttt{threads} means the number of threads per resource. The \texttt{ppr} is the number of CPU tasks per resource.
The last \texttt{<argument>} is command line arguments for the program, which is designed by programmer. 
In this case, I designed three type of arguments
\begin{enumerate}
  \item \texttt{-S, -s} Strategy, for specifying the pure mpi, hybrid 0 or 1 strategy.
  \item \texttt{-F, -f} file name, if this argument is defiend, the results will be stored in the file.
  \item \texttt{-H, -h} Helper message, the usage information.
  \item \texttt{-V, -v} Showing the version of program.
\end{enumerate}



\subsubsection{Computational Topology}
The computational topology is critically important when we are programming parallel PDEs solver softwares.
Put the strongly speed-dependent data into the slow memory could make entire program slower.

\paragraph{Cluster}
The cluster we are using for this project is \texttt{Callan} \cite{Callan_TCD} which has 
$2$ CPUs per compute node, and each CPU has $32$ cores with single thread. 
The Non-Uniform Memory Access (NUMA) nodes are layout as following 
\begin{figure}[htbp]
  \centering
  \begin{tikzpicture}[scale=1, transform shape]
    \node[anchor=south west,inner sep=0] (image) at (0,0) {\includegraphics[width=0.5\textwidth]{figure/FIG_Topology_9242.pdf}};
    % \node[anchor=south west,inner sep=0] (image) at (0,0) {\includegraphics[width=0.58\textwidth]{figure/FIG_Topology_9242.pdf}};     /// overleaf
    \begin{scope}[x={(image.south east)},y={(image.north west)}]
        % \draw[help lines, step=1em] (-1em,-1em) grid (30em,20em);    
        % % Draw axes
        % \draw[dashed,->] (-1em,0) -- (30em,0) node[right] {x};
        % \draw[dashed,->] (0,-1em) -- (0,20em) node[above] {y};

      \node at (12em,18em) {$2$ $Scokets$};

      \node at (4.5em,1em) {$DDR4$};
      \node at (19.5em,1em) {$DDR4$};

      \node at (4.5em, 15em) {$DDR4$};
      \node at (19.5em,15em) {$DDR4$};

      \node at (12em,13em) {$Hyper-threads$};
      \node at (12em,5.5em) {$Hyper-threads$};

      \node at (12em,11em) {$48$ $CPUS$};
      \node at (12em,3.5em) {$48$ $CPUS$};
    \end{scope}
  \end{tikzpicture}
  \caption{NUMA topology of single node on Cluster}
  \label{FIG_Topology_Callan}
\end{figure}
Accessing the other NUMA node's memory reduces the bandwidth and also the latency,
though the bandwidth is commonly high enough, the latency can increase by 
$~30\%$ to $~400\%$ \cite{NUMA_Latency_TCD}. 
This latency becomes dangerous when writing shared memory parallel programs.


\subsection{Comparison on single node}
On single node, the CPUs are connected by high-bandwidth, low-latency internal bus which is faster than connection between nodes.
However, for the $4$ total NUMA nodes per compute node, memory accessing between them has higher latency than cache.
Thus, the first tests set were run on the platform with single node, to evaluate the parallelistic performance of heat equation on 2 and 3 dimension spaces with 
3 strategies.


\subsubsection{Strong Scaling}
Figure \ref{FIG:Benchmark:PURE_MPI} visualizes the comparison of my proposed parallelistic program using pure MPI on two dimension space heat euqation with 
the number of CPUs and various problem scales.
Overall, the more CPUs brings more performance among all scales from $512^2$ to the $32'768^2$ but can not break the speedup limits.
Compared with large scale bigger than $4096^2$, the light problems has less speedup as the number of CPUs increases.
By seeing the trend of speedup ratios drop as the CPU gets more, 
the trend can be readily discovered which is the as the scale of problems gets larger, the latter it will have performance-dropping.
Once the problem size is large enough, ($4096^2$ and larger), the solvers can get the more benefits from more CPUs.

\begin{figure}[htbp]
  \centering
  \includegraphics[width=0.6\textwidth]{figure/FIG_Benchmark_pure_mpi.pdf}
  \caption{
    Comparison of speedup ratios of strong scaling tests of pure MPI parallelized program. 
    The investigated problem scales are the power of $2$, exponents ranging from $9$ to $15$.
    The number of CPUs are also set as power of $2$, with additional numbers $24$, $48$ and $96$ matched the topologies of CPU.
  }
  \label{FIG:Benchmark:PURE_MPI}
\end{figure}

On the other hand, I also include some unconventionaly number of CPUs in scaling tests such as $24$, $48$ and $96$ for comparison. and 
the results are also shown in the figure \ref{FIG:Benchmark:PURE_MPI}.
From this figure, it is hard to tell the difference of these where it ought to indicate some information about its NUMA structure.
This is because the MPI communication does not strongly effected by memory structure, while the hybrid does.
Figure \ref{FIG:Benchmark:Hybrid} shows the difference, the hybrid strategy brings lower performance with small scale across all CPUs and 
approximately identical in the large scale cases.
The most visible change in figure \ref{FIG:Benchmark:Hybrid_0} is that the speedup ratio of problem with scale $4096^2$ 
exceded the limits on $4$, $8$ and $16$ CPUs which is $1$, $2$, $4$ threads of each MPI process.
We can also see that when the number of threads is $8$ and $4$ MPI processes, the performances on large scale is better than pure MPI parallelism.

\begin{figure}[htbp]
  \centering
  \subfigure[No overlapping comm./comp.]
  {
    \includegraphics[width=0.47\textwidth]{figure/FIG_Benchmark_hybrid_0.pdf}
    \label{FIG:Benchmark:Hybrid_0}
  }
  \hfill
  \subfigure[With overlapping comm./comp.]
  {
    \includegraphics[width=0.47\textwidth]{figure/FIG_Benchmark_hybrid_1.pdf}
    \label{FIG:Benchmark:Hybrid_1}
  }
  \caption{
    Comparison of speedup ratios of strong scaling tests of mater-only parallelized program with overlapping and no overlapping of computation and communication.  
    The vague background is the results of pure MPI parallelization from figure \ref{FIG:Benchmark:PURE_MPI} and problems sclaes are identical as well.
    The number of threads are set to $1$, $2$, $4$, $8$, $16$, and $24$, 
    tasks per CPU are $1$, $2$ and $4$.
  }
  \label{FIG:Benchmark:Hybrid}
\end{figure}

For the other funneled hybrid parallelization, the figure 
            \ref{FIG:Benchmark:Hybrid_1} 
in the appendix shows the details of results,
this strategy has nearly idential performance of master only with no overlapping on large problem scales.
However, the behaviour of it on small scales has a different pattern.
This indicates that the overlappings of computation and communication are not as good as previous one, which means the overload management is not 
well on these tests.


\paragraph{Superliner Speedup}
Conventionally, the actuall speedup won't excedes the theoratical predictions of Amdalh's law.
However, the scaling of two hybrid programs did exceded the limits but exclusively on the problem scale $4096^2$ and $1$ to $8$ threads of each $4$ MPI processes.
Considering the details of the CPU used for these tests, 
\begin{itemize}
  \item It has $4$ NUMA node per CPU and once of which has $12$ CPUs with $2$ threads.
  \item It has \texttt{32KB} L1 data and L1 instruction cache, \texttt{1024KB} L2 cache and \texttt{36608KB} L3 cache.
\end{itemize}
The data type for this solver is \texttt{Double} which takes $8$ bytes, and $4096^2$ \texttt{Double} numbers takes 128 \texttt{MB} to store.
In the case of $4$ MPI processes, each process own a quater of number which uses 32 \texttt{MB} for handling sub-problems.
On the other hand, the L3 cache is 36608 \texttt{KB} = 35.75 \texttt{MB} which is just bigger than the sub-problem scale 32 \texttt{MB}.
When the problem size gets larger, such as $8096^2$ which takes 128 \texttt{MB} to store the sub-problem. 
In such case, the L3 case is no longer able to hold it, thus part of the numbers will be stored in the DDR4 memories which is lower bandwidth and higher latency than cache.
Moreover, due to the CPU enables hyper-threading, a NUMA node actually holds $12$ CPUs, which makes the 
superlinear speedup disappear when the threads is 16 and 24.

% \begin{figure}[htbp]
%   \centering
%   \includegraphics[width=0.7\textwidth]{figure/FIG_Benchmark_hybrid_1.pdf}
%   \caption{<caption>}
%   \label{FIG:Benchmark:Hybrid_1}
% \end{figure}

% \begin{figure}[htbp]
%   \centering
%   \subfigure[Grid sizes: 150*45*40/25*33*40.]{
%       \includegraphics[width=0.45\textwidth]{weak.png}
%       \label{fig:weak_scaling_1}    
%   }
%   \hfill
%   \subfigure[Grid sizes: 300*90*80/50*65*80]{
%       \includegraphics[width=0.48\textwidth]{weak_scaling_2_efficiency_comparison.png}    
%       \label{fig:weak_scaling_2}
%   }
%   \caption{
%       Weak scaling tests of Cartesian / Cylinder equation with same problem scale on $1$, $2^2$, $3^3$, $4^3$ processors.
%   }
%   \label{fig:weak_scaling}    
% \end{figure}

\subsubsection{Weak Scaling}
The weak scaling tests of an high performance program runing on cluster is necessary for researching the inner relationships 
between the number of CPUs and the scale of problems.
In order to get better weak scaling performance, the fraction of synchronization among processes and the cost of communications 
plays a minor role when the number of resources increase. 

Table \ref{TAB:Benchmark:Weak_PURE_MPI} lists the comparison of three different parallel strategies.
Overall, both three strategies have good weak scaling results across all problem sizes.
For example, according to the Gaustafsson's Law \ref{THEO:GaustafssonLaw}, the program using
overlapped strategy has sequential fraction 
$$f_s = \frac{49.989 - 64}{1-63} \approx 0.222$$
on the problem scale $4096^2$ with 64 CPUs
and $f_s \approx 0.147$ with 16 CPUs.

\begin{table}
  \caption{Weak Scaling on Single Node of 2D Heat Equation}
  \label{TAB:Benchmark:Weak_PURE_MPI}
  \begin{minipage}{\columnwidth}
    \begin{center}
      % \footnotesize % overleaf
      \begin{tabular}{>{\bfseries}p{3cm} p{1.5cm} p{1.5cm} p{1.5cm} p{1cm}}
        \toprule
        \multirow{2}{*}{Strategy}     & \multirow{2}{*}{\bfseries Size} & \multicolumn{3}{c}{\bfseries  Number of CPUs}   \\
                                      &                                  & \bfseries 4   & \bfseries 16   & \bfseries 64   \\
        \midrule
        Pure MPI      & \multirow{3}{*}{$512^2$}      & 4.006  & 12.497  & 47.849 \\
        No Overlap    &                               & 2.876  & 11.206  & 42.754 \\
        With Overlap  &                               & 3.173  & 10.818  & 42.282 \\
        \midrule
        Pure MPI      & \multirow{3}{*}{$1024^2$}     & 3.838  & 9.304   & 33.707 \\
        No Overlap    &                               & 3.947  & 12.995  & 33.447 \\
        With Overlap  &                               & 4.024  & 12.932  & 33.361 \\
        \midrule
        Pure MPI      & \multirow{3}{*}{$2048^2$}     & 2.376  & 8.245   & 31.203 \\
        No Overlap    &                               & 3.874  & 8.972   & 31.510 \\
        With Overlap  &                               & 3.740  & 8.989   & 31.430 \\
        \midrule
        Pure MPI      & \multirow{3}{*}{$4096^2$}     & 3.543  & 8.245   & 31.203 \\
        No Overlap    &                               & 3.953  & 13.799  & 49.515 \\
        With Overlap  &                               & 3.948  & 13.800  & 49.989 \\
        \bottomrule
      \end{tabular}
    \end{center}
    % \bigskip
    % \footnotesize\emph{Source:} This is source 
  \end{minipage}
\end{table}



\subsection{Comparison on multi-node}
\subsection{Comparison}

% \subsection{Finite Difference Methods}
% \subsubsection{Pure Message Passing Parallel}
% \subsubsection{Hybrid Parallel}


% \subsection{Physics Informed Neural Networks}
% \subsubsection{CUDA parallel}
% \subsubsection{Hybrid Parallel}


\subsection{Visualization}












% Table 
% lists the results 

% \begin{table}
%   \caption{Strong Scaling on Single Node of 2D Heat Equation}
%   \label{}
%   \begin{minipage}{\columnwidth}
%     \begin{center}
%       \begin{tabular}{lcccccc}
%         \toprule
%         Scale & $1024^2$ & $2048^2$ & $4096^2$  & $8192^2$  & $16384^2$ & $32768^2$\\
%         \midrule
%         2     & 1        & 1        &   1       & 1         & 1         & 1 \\
%         4     & 1        & 1        &   1       & 1         & 1         & 1 \\
%         8     & 1        & 1        &   1       & 1         & 1         & 1 \\
%         16     & 1        & 1        &   1       & 1         & 1         & 1 \\
%         32     & 1        & 1        &   1       & 1         & 1         & 1 \\
%         48     & 1        & 1        &   1       & 1         & 1         & 1 \\
%         64     & 1        & 1        &   1       & 1         & 1         & 1 \\
%         96     & 1        & 1        &   1       & 1         & 1         & 1 \\
%         \bottomrule
%       \end{tabular}
%     \end{center}
%     % \bigskip
%     % \footnotesize\emph{Source:} This is source 
%   \end{minipage}
% \end{table}

\section{Experiments}
\section{Conclusion}
\section{Acknowledgement}
\section{Discussion}
The major part of this project 
is to implement a efficient FDM PDE solver 
on parallel systems and validate the 
quality of the performance on accuracy, scaling.
Due to the time constrains, 
as well as the lack of hardware, software and financial support,
I have not do full profiling of the FDTD and PINN models but only 
partial analysis using \texttt{gperftools} \cite{gperftools}.


\subsection{Further Work}
Though out the FDTD method, 
various of \texttt{C++} features are used for designed a safer and faster library,
also a number of MPI concepts have been used along with the \texttt{C++}, \texttt{OpenMP} features
to designed a fast, efficient and user-friendly implementation.
Also, a variety of \texttt{C++} concepts and \texttt{Libtorch} concepts are used for implementing 
a fast PINN solver for PDEs.

However, this project provides precise, efficient and fast PDE solver for cluster environment,
there are still many things need to do.


% \subsection{Memory}
\paragraph{Memory}
FDTD is implemented by ping-pong update strategy which is designed for higher cache hit rate.
However, for some system, the memory is limited resource for storing doubled size of the required solutions.
In such case, using single matrix to store the grid and updating it using 
Gauss-Seidel or red-black strategy can leads more memory efficiency.

% \subsection{Workload Management}
\paragraph{Workload Management}
Although the current workload is managed by OpenMP \texttt{singe}, \texttt{barrier}, \texttt{critical} and others.
It's still hard to know that exact workload for each distributed CPUs.
In general, using \texttt{sections}, \texttt{tasks} can make it better.
However, in latter version MPI-3.X, it support MPI shared memory programming without 
the worries of OpenMP threads. 
Thus, using Hybrid MPI + MPI with shared memory nodes can have better performance.

% \subsection{PDEs}
\paragraph{PDEs}
Due to the time constrainment, I had not implement the full weak/strong scaling testing of Von Neumman Boundary Conditions.
In further work, applying the different type of boundary conditions efficiently is also worth to do.
Also, the domain of PDE in this project was set to the $[0,1]^d$ in d-dimension space, which is not likely happen in real case.
Thus, implementing an efficient mesh generator, and mesh-based FDM method is an other work direction.

% \subsection{PINN}
\paragraph{PINN}
Implementing neural network in \texttt{C++}
allows us have better performance.
Also, parallel training of neural network on GPU is implemented by MPI.
Thus, using MPI and Libtorch to implement a parallelized neural network training program is also valuable 
as a comparison for numerical methods.


\subsection{Conclusion}
In this project, the ideas were creating a versatile library which includes 
a high efficient matrix library that can be used in parallelized environment, 
different MPI parallel environment such as Cartesian topology, Cylinder topology etc., 
a generalized user-friendly PDE solver based on FDM algorithm which allows user specifying 
the initial conditions, the type of boundary conditions and the parallel strategies such a pure MPI or hybrid of MPI/SMP.

The first is to implement of template multi-dimension array object is created with deep features 
which have direct access to memory.
Subsequently, a user interface template object of multi-dimension array created for specific features which will be used 
in latter FDTD and PINN models.
On the other, an environment of MPI communication object was build for save initializing and finalizing MPI environment.
Moreover, MPI Cartesian Topology was integrated into a object aligned with other 
features for distributing template multidimensional array.
Enhance, defining distributed IO based on template array called \texttt{gather}, MPI-IO is also constructed for 
parallel IO which is designed for saving/loading large scale problems without communication and saving memory usage.

The FDTD models are implemented in three ways, pure MPI parallelism, master only with no computation / communication overlapping 
and funneled master only with overlap of communication and computation. 
Also provide choice of two type of boundary condition, Dirichlet and Von Neumman Boundary Condition.
A large proportion of this project is to build the library and running weak/strong scaling test to determine the differences between
three strategies.
In general, on single node or on single chip, pure MPI has similar performance comparing to OpenMP.
On multi-nodes, with a reasonable resources allocation, the hybrid strategies have better performance than pure MPI.
These results also indicate that managing the overload of each processes is difficult.

The neural network is the major role of the state-of-art comparison.
This project avoid the conventional approach for implementing neural networks using \texttt{Python} by using \texttt{C++}.
For comparing the performance, efficiency, I constructed 
a neural network with modified loss function and the training strategy in \texttt{C++} which is called Physics Informed Neural Network.

Overall, this projects provides a polymorphic-library of finite difference methods, 
a fast and stable parallel FDTD PDE solver with three type parallelization for different scenarios, 
a comparison between one of the most popular simulation tool, PINN.






% \bibliographystyle{ACM-Reference-Format}
% \bibliography{reference}

\printbibliography

\appendix

\end{document}