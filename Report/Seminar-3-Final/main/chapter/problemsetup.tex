\section{Problem Setups}

\begin{frame}
  \frametitle{Statement}
  \framesubtitle{Challanges \& Objectives}
  \begin{block}{Challanges of Nonlinear Multi-Scale Systems}
    \begin{enumerate}
      \item Difficulty in modeling and predicting inhomogeneous cascades of scales.
      \item High computational costs and uncertainties in classical methods.
    \end{enumerate}    
  \end{block}

  \begin{block}{Project Objectives}
    \begin{enumerate}
      \item Implement FDTD and PINN in \texttt{C++/C}.
      \item Implement FDTD hybrid parallel version using \texttt{MPI/OpenMP}.
      \item Implement PINN GPU parallel version using \texttt{Libtorch/CUDA}.
      \item Optimize performance through parallel computing frameworks.
      \item Evaluate the efficiency and accuracy of FDTDs and PINNs.
    \end{enumerate}
  \end{block}
\end{frame}


\begin{frame}
  \frametitle{General Form of problem}
  \begin{block}{}
    The PDE parametrized by number $\lambda$ and an operator $\mathcal{N}[\cdot; \lambda]$, 
    and assume the variable $x$ is a 2D or 3D spatio-vector which is written in 
    \begin{equation}
      \begin{cases}
        \displaystyle \frac{\partial u}{\partial t}\left(t,\vec{x}\right) + \mathcal{N}\left[u;\lambda\right] = 0 \\
        \displaystyle u\left(0,\vec{x}\right) = \varphi (\vec{x})
      \end{cases}
    \end{equation}
    where $\varphi$ is the initial condition, and $\vec{x}\in \Omega, t\in[0, +\infty)$.
  \end{block}



  \begin{block}{Boundary Conditions}
    The Dirichlet and Von Neurmann boundary conditions are formed as 
    \begin{equation}
      \begin{cases}
        \displaystyle u\left(t,\vec{x}\right) = g (t,\vec{x}) \\
        \displaystyle \frac{\partial u}{\partial \vec{n}} = g (t,\vec{x})  
      \end{cases}
    \end{equation}
    where $\vec{n}$ is the normal vector on $\overline{\Omega}$ the boundary of domain $\Omega$.
  \end{block}
\end{frame}

\begin{frame}
  \frametitle{Heat Equation}
  \framesubtitle{Specific Example}
  \begin{block}{Heat Equation 2D}
    The function $u(t,x,y) = x + y - xy$, $\forall \alpha \in \mathbb{R}^1 $, is the solution of 2D Heat Equation \ref{EQ:Heat2D} below
    \begin{align}\label{EQ:Heat2D}
      &\frac{\partial u}{\partial t} = \alpha \left(
        \frac{\partial u^2}{\partial^2 x}
        +
        \frac{\partial u^2}{\partial^2 y}
      \right) &(x,y) \in \Omega, \: t \in \left[0, +\infty\right)  \nonumber\\
      &u(0,x,y)  = \varphi(x,y) = 0 &(x,y) \in \Omega\\
      &u(t,x,y)
       = g(x,y)
       = \begin{cases}
        y, \:\: x=0, y\in\left(0,1\right)\\
        1, \:\: x=1, y\in\left(0,1\right)\\
        x, \:\: y=0, x\in\left(0,1\right)\\
        1, \:\: y =1, x\in\left(0,1\right)
      \end{cases}
      &t \in \left[0, +\infty\right) \nonumber
    \end{align}
  \end{block}
\end{frame}


\begin{frame}
  \frametitle{Heat Equation}
  \framesubtitle{Specific Example}
  \begin{block}{Heat Equation 3D}
    The function $u(t,x,y,z) = x + y + z - 2xy - 2xz - 2yz + 4xyz$, $\forall \alpha \in \mathbb{R}^1 $, is the solution of 3D Heat Equation \ref{EQ:Heat3D} below
    \begin{align}\label{EQ:Heat3D}
      &\frac{\partial u}{\partial t} = \alpha \left(
        \frac{\partial u^2}{\partial^2 x}
        +
        \frac{\partial u^2}{\partial^2 y}
        +
        \frac{\partial u^2}{\partial^2 z}
      \right) & (x,y, z) \in \Omega, \: t \in \left[0, +\infty\right) 
                                                                      \nonumber\\
      &u(0,x,y,z)  = \varphi(x,y,z) = 0 &(x,y,z) \in \Omega\\
      &  u(t,x,y,z) = g(x,y,z) = 
      \begin{cases}
        y+z -2yz        , \:\: &x=0,\\
        1 - y - z + 2yz , \:\: &x=1,\\
        x+z - 2xz       , \:\: &y=0,\\
        1 - x - z + 2xz , \:\: &y =1,\\
        x+y - 2xy       , \:\: &z=0,\\
        1 - x - y + 2xy , \:\: &z=1
      \end{cases}
      &t \in \left[0, +\infty\right) \nonumber
    \end{align}
  \end{block}
\end{frame}

% \subsection{Continuous Form}
% \subsection{Discretization}

