\section{Discussion}



\begin{frame}
  \frametitle{Conclusion}
  \framesubtitle{Finite Difference Time Domain Method}
  For 2D and 3D Thermal Conduction PDE systems \\
  I implemented the FDTD methods in fined region with three parallel models:
  \begin{itemize}
    \item Pure MPI model 
    \begin{enumerate}
      \item had best performance on single compute node overall.
      \item speedup ratio quickly drop on multi-node.
    \end{enumerate}
    \item MPI/OpenMP Hybrid models 
    \begin{enumerate}
      \item had lower performance in general.
      \item can make more use of the advantage of L3 cache.
      \item superliner speedup in certain scenario.
    \end{enumerate}
  \end{itemize}

\end{frame}


\begin{frame}
  \frametitle{Conclusion}
  \framesubtitle{Physics Informed Neural Network}
  For 2D and 3D Thermal Conduction PDE systems \\
  I implemented PINN models and trained on generated datasets, comparing with FDTD, the PINN 
  \begin{enumerate}
    \item had identical or higher accuracy.
    \item had less memory usage.
    \item took shorter time to get the predictions(results).
    \item had more flexibility on producing the results.
  \end{enumerate}


\end{frame}


\section{Further Research Directions}
\begin{frame}
  \frametitle{Further Research Directions}
  \begin{itemize}
    \item Resource Management
    \item Workload Management
    \item MPI/CUDA Hybrid parallel of FDTD/PINN
    \item Other PDE Systems
  \end{itemize}
\end{frame}