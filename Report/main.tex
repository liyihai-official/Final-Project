% \documentclass[manuscript,screen,review]{acmart}
\documentclass[sigconf, screen, final, language=english, natbib=false]{acmart}
% \documentclass[manuscript, screen, review, language=english, natbib=false]{acmart}

\usepackage{algorithm, algorithmic}
\usepackage{listings}
\usepackage{color} % use color

\definecolor{dkgreen}{rgb}{0,0.6,0}
\definecolor{gray}{rgb}{0.5,0.5,0.5}
\definecolor{mauve}{rgb}{0.58,0,0.82}

\lstset{frame=tb,
  language=C++, % C++
  aboveskip=3mm,
  belowskip=3mm,
  showstringspaces=false,
  columns=flexible,
  basicstyle={\small\ttfamily},
  numbers=left,
  numberstyle=\tiny\color{gray},
  keywordstyle=\color{blue},
  commentstyle=\color{dkgreen},
  stringstyle=\color{mauve},
  breaklines=true,
  breakatwhitespace=true,
  tabsize=3
}

\usepackage{tikz}

%% \BibTeX command to typeset BibTeX logo in the docs
\AtBeginDocument{%
  \providecommand\BibTeX{{%
    \normalfont B\kern-0.5em{\scshape i\kern-0.25em b}\kern-0.8em\TeX}}}


\setcopyright{none}

\copyrightyear{2024}
\acmYear{2024}
\acmDOI{}
\acmISBN{}

\acmConference[MAP55640]{Final Project}{May 25, 2024}{Dublin, Ireland}

% Bibliography style
\RequirePackage[
  datamodel=acmdatamodel,
  style=acmnumeric,
  ]{biblatex}

% %% Declare bibliography sources (one \addbibresource command per source)
\addbibresource{reference.bib}









\begin{document}
\title{MAP55640 Final Project}
\author{Li Yihai}
\email{liy35@tcd.ie}
% \authornote{All codes are coded individually.}
\affiliation{%    
    \institution{Mathematics Institute}
    \institution{High Performance Computing}
    \city{Dublin}
    \country{Ireland}
}

\settopmatter{printacmref=false}


\begin{abstract}
\end{abstract}
    
% Keywords. The author(s) should pick words that accurately describe
% the work being presented. Separate the keywords with commas.
\keywords{
    Keywords
}

%%
%% This command processes the author and affiliation and title
%% information and builds the first part of the formatted document.
\maketitle

% ------------------------------------------------------------------------------------------------
\section{Introduction}
Numerical methods of solving partial differential equations (PDE) have 
demonstrate far better performance than many other methods such as finite 
difference methods (FDM) \cite{},
finite element methods (FEM) \cite{}, 
Lattice Boltzmann Method (LBM) \cite{}
and Monte Carlo Method (MC) \cite{}.
In recent years, researchers in the field of deep learning have mainly focused 
on how to develop more powerful system architectures and learning methods such 
as convolution neural networks (CNNs) \cite{}, 
Transformers \cite{} 
and Perceivers \cite{} .
In addition, more researchers have tried to develop more powerful models specifically 
for numerical simulations. 
Despite of the relentless progress, modeling and predicting the evolution of nonlinear 
multiscale systems which has inhomogeneous cascades-scales by using classical analytical 
or computational tools inevitably encounts severe challanges and comes with prohibitive
cost and multiple sources of uncertainty.

\section{Related Work}
% 在工程学、物理学、生物学和化学等众多领域,为了解决各种各样的偏微分方程,学术界尝试了非常多的途径,诞生了很多数值解法并且都已经非常成熟。
% 有限差分法是一种被广泛运用的方法,通过差商来近似所需要的连续函数的导数。

To gain well quality solution of various types of PDEs is prohibitive and notoriously challanging.
The number of methods avaliable to determine canonical PDEs is limited as well,
includes 
separation of variables, 
superposition, 
product solution methods, 
Fourier transforms, 
Laplace transforms and 
perturbation methods, 
among a few others.
Even though there methods are exclusively well-performed on constrained conditions,
such as regular shaped geometry domain, constant coefficients, well-symmetric conditions 
and many others.
These limits strongly constrained the range of applicability of numerical techniques for solving PDEs,
rendering them nearly irrelevant for solving problems pratically.

General, the methods of determining numerical solutions of PDEs can be broadly classified into
two types: 
deterministic 
and stochastic. 
The mostly widely used stochastic method for solving PDEs is 
Monte Carlo Method \cite{Monte Carlo Method} which is a popular method in solving PDEs in higher dimension space with 
notable complexity.

\subsection{Finite Difference Method}
The Finithe Difference Method(FDM) is based on the numerical approximation method in calculus of finite differences.
The motivation is quiet straightforward which is approximating solutions by finding values satisfied PDEs on a set of 
presctibed interconnected points within the domain of it. Those points are which referd as nodes, and the set of nodes 
are so called as a grid of mesh.
A notable way to approximate derivatives are using 
Taylor Series expansions.
Taking 2 dimension Possion Equation as instance, assuming the investigated value as, $\varphi$,
\begin{equation}\label{EQ_POSSION_2D}
  \frac{
    \partial ^ 2 \varphi  
  }{
    \partial x ^ 2
  } +
  \frac{
    \partial ^ 2 \varphi  
  }{
    \partial y ^ 2
  }
  =  f(x,y)
\end{equation}
The total amount of nodes is denoted with $N = 15$, which gives the numerical equation which governing 
equation \ref{EQ_POSSION_2D} 
shown in 
equation \ref{EQ_POSSION_2D_10_NODES} and nodes layout as shown in the 
figure \ref{FIG_POSSION_2D_10_NODES}
\begin{equation}\label{EQ_POSSION_2D_10_NODES}
  \frac{
    \partial ^ 2 \varphi_i
  }{
    \partial x_i ^ 2
  } +
  \frac{
    \partial ^ 2 \varphi_i
  }{
    \partial y_i ^ 2
  }
  =  f(x_i,y_i) = f_i, \:\:\:\:\: i = 1,2,\dots,15
\end{equation}
\begin{figure}[htbp]
  \centering
  
  \caption{The Schematic Representa of of a 2D Computiatioal Domain and Grid. The nodes are used for the FDM by solid circles. 
  Nodes $11-15$ denote boundary nodes, while nodes $1-10$ denote internal nodes.}
  \label{FIG_POSSION_2D_10_NODES}
\end{figure}
In this case, we only need to find the value of internal nodes which $i$ is ranging from $1$ to $10$.
Next is aimming to solve this linder system \ref{EQ_POSSION_2D_10_NODES}.


\subsection{Physics Informed Neural Networks}
With the explosive growth of avaliable data and computing resouces, 
recent advances in machine learning and data analytics have yieled good results across science discipline, 
including Convolutional Neural Networks (CNNs) \cite{CNN}
for image recoginition, 
Generative Pre-trained Transformer (GPT) \cite{GPT}
for natual language processing and 
Physics Informed Neural Networks (PINNs) \cite{PINN}
for handling science problems with high complexity.
PINNs is a type of machine learning model makes full use of the benefits from 
Auto-differentiation (AD) \cite{AD}
which led to the emergence of a subject called 
Matrix Calculus \cite{Matrix_Calculus}.
Considering the parametrized and nonlinear PDEs of the general form [E.q. \ref{EQ_General_PDEs}] of function $u(t,x)$
\begin{equation}\label{EQ_General_PDEs}
  u_t + \mathcal{N}\left[u;\lambda\right] = 0
\end{equation}
The $\mathcal{N}[\cdot;\lambda]$ is a nonlinear operator which parametrized by $\lambda$.
This setup includes common PDEs problems like heat equation, and black-stokz equation and others.
In this case, we setup a neural network $NN[t,x;\theta]$ which has trainable weights $\theta$ and takes 
$t$ and $x$ as inputs, outputs with the predicting value $\hat{u}(t,x)$.
In the training process, the next step is calculating the necessary derivatives of $u$ with the respect to $t$ and $x$.
The value of loss function is a combination of the metrics of how well does these predictions fit the given conditions and 
fit the natural law [Fig. \ref{FIG_Schematic_View_PINN}]. 
\begin{figure}[htbp]
  \centering
  
  \caption{The Schematic Representa of a structure of PINN.}
  \label{FIG_Schematic_View_PINN}
\end{figure}
\section{Problem Setups}
In this project, I chose to use various numerical approaches to approximate the solutions.
It begins with Finite Difference Methods(FDM) and an other method employed by deep neural 
networks which leverage by their capability as universal function approximators
\cite{DNN-HORNIK1989359}. 

The Kormoglov PDEs are series of equations which describe the motions of Brownian Motions
\cite{SolveKorPDE}.
In general, 
let $T \in (0, +\infty)$, $d \in \mathbb{N}$, 
let $\mu : \mathbb{R}^d \rightarrow \mathbb{R}^d$ 
and $\sigma: \mathbb{R}^d \rightarrow \mathbb{R}^{d \times d}$ be the Lipschitz continuous fucntions.
Let $\varphi : \mathbb{R}^d \rightarrow \mathbb{R}$ be a function, 
and $u$ be a function from Hilbert Space 
$[0, T] \times \mathbb{R}^d$ to $\mathbb{R}$
\begin{align*}
  u : [0, T] \times \mathbb{R}^d & \longrightarrow \mathbb{R} \\
      (t, x) & \longmapsto u 
\end{align*}
with at most polynomially growing partial derivatives.
The problem is that $u$ satisfied a below system on $D = [0, T] \times \left[a, b \right]^d $ and 
$a, b \in \mathbb{R}^d$ with $a < b$,
\begin{align}
  & \frac{\partial u}{
    \partial t} 
  = \frac{1}{2}
  \text{Trace}_{\mathbb{R}^d}
  \left[
    \sigma(x) \sigma^{*}(x)
    \left(\text{Hess}_x u\right)
  \right]
  +
  \langle 
    \mu, 
    \nabla _x u
  \rangle_{
    \mathbb{R}^d
  } \\
 & u(0, x) = \varphi(x), 
 \:\:\:\: 
 x \in \left[a, b \right]^d
 \\
 & \left.
 \left(
  \frac{\partial u}{\partial \vec{n}}  + \lambda u 
 \right)
 \right|_{x \in \varGamma}
 = g(t, x) 
 \:\:\:\: 
 \forall t \in [0, T],\: x \in \mathbb{R}^d
\end{align}

The goal is to numerically appriximate the stablized state $u(T, x)$ of the system in the fureture time $T$,
and in the hypercude space $\left[a, b\right]\in \mathbb{R}^d$ with various boundary conditions $g$.

Reprompt problem, this work aimed at solving this problem in a bigger picture which requires a more general 
form. 
In this project, I consider the parametrized and nonliner PDEs of the general form
\begin{align}
  \frac{\partial u}{\partial t}
  + \mathcal{N} 
  \left[
    u; \lambda
  \right] 
  = 0
  \:\:\:\: 
  t \in \left[0. T\right], x \in D
\end{align}
where $\mathcal{N}\left[\cdot; \lambda \right]$ stands for a nonlinear operator parametrized by $\lambda$.

\subsection{Navior-Stokz Equation}
\section{Methodology}
\section{Experiments}
\section{Conclusion}
\section{Acknowledgement}



% \bibliographystyle{ACM-Reference-Format}
% \bibliography{reference}

\printbibliography

\appendix

\end{document}