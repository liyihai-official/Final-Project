% \documentclass[manuscript,screen,review]{acmart}
% \documentclass[sigconf,screen, final]{acmart}
\documentclass[manuscript, screen, review, language=english]{acmart}

\usepackage{algorithm, algorithmic}
\usepackage{listings}
\usepackage{color} % use color

\definecolor{dkgreen}{rgb}{0,0.6,0}
\definecolor{gray}{rgb}{0.5,0.5,0.5}
\definecolor{mauve}{rgb}{0.58,0,0.82}

\lstset{frame=tb,
  language=C++, % C++
  aboveskip=3mm,
  belowskip=3mm,
  showstringspaces=false,
  columns=flexible,
  basicstyle={\small\ttfamily},
  numbers=left,
  numberstyle=\tiny\color{gray},
  keywordstyle=\color{blue},
  commentstyle=\color{dkgreen},
  stringstyle=\color{mauve},
  breaklines=true,
  breakatwhitespace=true,
  tabsize=3
}

\usepackage{tikz}

%% \BibTeX command to typeset BibTeX logo in the docs
\AtBeginDocument{%
  \providecommand\BibTeX{{%
    \normalfont B\kern-0.5em{\scshape i\kern-0.25em b}\kern-0.8em\TeX}}}


\setcopyright{none}

\copyrightyear{2024}
\acmYear{2024}
\acmDOI{}
\acmISBN{}

\acmConference[MAP55640]{Final Project}{May 25, 2024}{Dublin, Ireland}

\begin{document}
\title{MAP55640 Final Project}
\author{Li Yihai}
\email{liy35@tcd.ie}
% \authornote{All codes are coded individually.}
\affiliation{%    
    \institution{Mathematics Institute}
    \institution{High Performance Computing}
    \city{Dublin}
    \country{Ireland}
}

\settopmatter{printacmref=false}


\begin{abstract}
\end{abstract}
    
% Keywords. The author(s) should pick words that accurately describe
% the work being presented. Separate the keywords with commas.
\keywords{
    Keywords
}

%%
%% This command processes the author and affiliation and title
%% information and builds the first part of the formatted document.
\maketitle

% ------------------------------------------------------------------------------------------------
\section{Introduction}
Numerical methods of solving partial differential equations (PDE) have 
demonstrate far better performance than many other methods, such as finite 
difference methods (FDM) \cite{}, 
finite element methods (FEM) \cite{}, 
Lattice Boltzmann Method (LBM) \cite{}
and Monte Carlo Method (MC) \cite{}.
In recent years, researchers in the field of deep learning have mainly focused 
on how to develop more powerful system architectures and learning methods such 
as convolution neural networks (CNNs) \cite{}, 
Transformers \cite{} 
and Perceivers \cite{} .
In addition, more researchers have tried to develop more powerful models specifically 
for numerical simulations.

\section{Related Work}
\section{Problem Statement}
\section{Methodology}
\section{Experiments}
\section{Conclusion}
\section{Acknowledgement}

\appendix

\end{document}